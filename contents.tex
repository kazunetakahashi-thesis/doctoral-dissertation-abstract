%
% File    : contents.tex
% Author  : Kazune Takahashi
% Created : 1/5/2019, 8:59:35 PM
% Powered by Visual Studio Code
%

\section{主定理}

本論文では, $2$ 種類の臨界 Sobolev 指数を持つ Hénon 型方程式を考察する.
$N \geq 3$ を自然数とし, $2^* = 2N/(N-2)$ を臨界 Sobolev 指数とする.
$\Omega \subset \R^N$ を区分的 $C^1$ 級有界領域とし,
$\Omega \subset B(0, 1)$ を充たすものとする.
ここで, $B(x, r) = \{ y \in \R^N \mid \lvert y - x \rvert < r \}$
である.
$x_0 = (1, 0, \dots, 0) \in \R^N$ とする.
$x_0 \in \partial \Omega$ 及び
$\Omega$ が $x_0$ で内部球条件を充たすことを仮定する.
関数 $\Psi \in L^\infty(\Omega) \setminus \{ 0 \}$ について,
以下の $2$ つの条件を導入する.
\begin{enumerate}[(T1)]
  \setcounter{enumi}{-1}
  \item \label{enum:T0}
    $\Psi \geq \kappa_0 \tin \omega$ を充たす
    開集合 $\omega \subset \Omega$ と
    定数 $\kappa_0 > 0$ が存在する.
  \item \label{enum:T1}
    $m > 0$, $\beta \geq 0$ 及び開球
    $B \subset \Omega$ が存在し,
    $x_0 \in \partial \Omega$ 及び
    $\Psi \geq \Psi_0 \tin \Omega$ が成立する.
    ここで, $\Psi_0$ は以下で定める.
    \begin{equation}
      \Psi_0(x) =
      \begin{cases}
        m \left\lvert x - x_0 \right\rvert^\beta
        & x \in B, \\
        0 & x \not\in B.
      \end{cases}
      \label{eq:Psi_0}
    \end{equation}
\end{enumerate}

本論文の Chapter 3 では以下の方程式について考察する.
\begin{align}
  \left\{
  \begin{aligned}
    - \Delta u & = \lambda \Psi u + \lvert x \rvert^\alpha u^{2^*-1}
               &                                                     & \tin \Omega,                        \\
    u          & > 0                                                 &              & \tin \Omega,         \\
    u          & = 0                                                 &              & \ton \partial\Omega.
  \end{aligned}
  \right. \label{eq:henon_prob_main}
\end{align}
定数 $\lambda$ は $\lambda < \lambda_1$ を充たすものとする.
ここで $\lambda_1$ は,次の
Dirichlet 固有値問題の第 $1$ 固有値とする:
$-\Delta \phi = \lambda \Psi \phi \tin \Omega$.
また, $\alpha > 0$ とする.

\begin{thm}[Theorem 1.1.1] \label{thm:henon_main}
  $N \geq 4$, $0 < \lambda < \lambda_1$ とする.
  $\Psi \in L^\infty(\Omega) \setminus \{ 0 \}$
  は $0 \leq \Psi \leq 1 \tin \Omega$ を充たすものとする.
  (T\ref{enum:T1}) を仮定する.
  このとき,十分小さな $\alpha > 0$ に対し,
  方程式 \eqref{eq:henon_prob_main}
  は解 $u \in H_0^1(\Omega)$ を持つ.
\end{thm}

本論文の Chapter 4 では以下の Kirchhoff 型方程式について考察する.
\begin{align}
  \left\{
  \begin{aligned}
    - \left( a + b \left( \int_\Omega \lvert Du \rvert^2 dx \right)^{(p-2)/2} \right) \Delta u &= \Psi u^{q-1} + \lvert x \rvert^\alpha u^{2^* - 1}
               &                                                     & \tin \Omega,                        \\
    u          & > 0                                                 &              & \tin \Omega,         \\
    u          & = 0                                                 &              & \ton \partial\Omega.
  \end{aligned}
  \right. \label{eq:kirchhoff_prob_main}
\end{align}
ここで $\alpha \geq 0$, $p > 2$, $q \geq 2$, $a \geq 0$, $b \geq 0$
は $a + b > 0$ を充たす定数である.

\begin{thm}[Theorem~1.1.2] \label{thm:kirchhoff_main}
  $2 < p < q < 2^*$ とする.
  $\Psi \in L^\infty(\Omega) \setminus \{ 0 \}$ は,
  ある定数 $\kappa > 0$ が存在し
  $0 \leq \Psi \leq \kappa \tin \Omega$
  を充たすものとする.
  以下の条件 (1), (2) のいずれかの成立を仮定する.
  \begin{enumerate}[(1)]
    \item $N = 3$ and $4 < q < 2^* = 6$,
    \item $N \geq 4$.
  \end{enumerate}
  \begin{enumerate}[(i)]
    \item
      $\alpha = 0$ とする. (T\ref{enum:T0}) を仮定する.
      このとき,方程式 \eqref{eq:kirchhoff_prob_main}
      は解 $u \in H_0^1(\Omega)$ を持つ.
    \item
      $\alpha > 0$ とする. (T\ref{enum:T1}) を仮定する.
      このとき,十分小さな $\alpha > 0$ に対し,
      方程式 \eqref{eq:kirchhoff_prob_main}
      は解 $u \in H_0^1(\Omega)$ を持つ.
  \end{enumerate}
\end{thm}

(T\ref{enum:T1}) の条件を充たす例として,以下の系がある.
この例では, $\Psi$ は $\partial \Omega$ で消えている.

\begin{cor}[Corollary~1.1.3] \label{cor:cor_of_main_theorem}
  $\Omega = B(0, 1)$, $\beta_0 > 0$ とする.
  $\Psi(x) = \left(1 - \lvert x \rvert \right)^{\beta_0}$ と定める.
  \begin{enumerate}[(i)]
    \item
      $N \geq 4$, $0 < \lambda < \lambda_1$ とする.このとき,
      小さな $\alpha > 0$ に対し, \eqref{eq:henon_prob_main} は解を持つ.
    \item
      $2 < p < q < 2^*$ とする.
      定理~\ref{thm:kirchhoff_main} の (1), (2) のいずれかの成立を仮定する.
      このとき,
      小さな $\alpha > 0$ に対し, \eqref{eq:kirchhoff_prob_main} は解を持つ.
  \end{enumerate}
\end{cor}

\section{先行研究}

Hénon 型方程式は~\cite{henon1973numerical} において研究が始まった.
~\cite{MR674869} を始め,数多くの研究がなされている.
また, $\alpha = 0$, $\Psi \equiv 1 \tin \Omega$ のとき,
\eqref{eq:henon_prob_main} は Brézis--Nirenberg
方程式~\cite{MR709644}となる.
Brézis--Nirenberg 型方程式は $30$ 年以上研究されている.
\eqref{eq:henon_prob_main} は Hénon 型方程式と
Brézis--Nirenberg 型方程式を併せたものと捉えられる.
~\cite{MR2951742} と~\cite{MR2951722} において,
\eqref{eq:henon_prob_main} と直接関係する以下の方程式が考察されている.
\begin{align}
  \left\{
  \begin{aligned}
    - \Delta u       & = \lambda u + \lvert x \rvert^\alpha \lvert u
    \rvert^{2^*-2} u &                                               & \tin \Omega,                         \\
    u                & = 0                                           &              & \ton \partial \Omega.
  \end{aligned}
  \right. \label{eq:henon-brezis-nirenberg}
\end{align}
$\alpha > 0$, $\lambda > \lambda_1'$ は定数である.
ここで $\lambda_1'$ は,
次の Dirichlet 固有値問題の第 $1$ 固有値とする:
$-\Delta \phi = \lambda \phi \tin \Omega$.
~\cite{MR2951742} 及び~\cite{MR2951722} では,
それぞれ,
$N \geq 7$ かつ $\partial\Omega$ が滑らかなとき,及び,
$N \geq 5$ かつ $\Omega = B(0, 1)$ のとき,
\eqref{eq:henon-brezis-nirenberg} が符号変化解を
小さな $\alpha > 0$ に対して持つことが示されている.
本論文の Chapter 3 では $0 < \lambda < \lambda_1$ のときの
正値解について, $N \geq 4$ かつ $\Omega$ がより一般的であり,
$\Psi$ が定数とは限らない場合を考察する.

$p = 4$, $\alpha = 0$, $\Psi \equiv \lambda$ が定数のとき,
\eqref{eq:kirchhoff_prob_main} は標準的な,
臨界 Sobolev 指数を持つ Kirchhoff 型の楕円型方程式となる.
~\cite{MR3278854} においては,
$N = 3$, $4 < q < 6$ のとき,
全ての $\lambda > 0$ に対し解が存在することが示されている.
また,~\cite{MR3018020} 及び~\cite{MR3278854} においては,
$N = 3$, $2 < q \leq 4$ のとき,
ある $\lambda^* \geq 0$ が存在し,
全ての $\lambda > \lambda^*$ に対し解が存在することが
それぞれ異なる汎関数の切り捨て手法により示されている.
これらの研究に触発され,
臨界 Sobolev 指数を持つ Kirchhoff 型の楕円型方程式
は近年活発に研究がなされている.
$N = 4$ での研究結果としては~\cite{MR3210026} が挙げられる.
また,非局在項の拡張もなされている.
特に~\cite{MR3539075} では,定理~\ref{thm:kirchhoff_main}~(i) が
$a > 0$ の場合に示されている.
\eqref{eq:kirchhoff_prob_main}
は臨界 Sobolev 指数を持つ Kirchhoff 型方程式
と Hénon 型方程式を併せたものと捉えられる.
本論文の Chapter 4 で
\eqref{eq:kirchhoff_prob_main} を考察するが,この章の目的の 1 つは,
Chapter 2 で得られる結果を~\cite{MR3278854} の議論に応用し,
非局在項をより一般化した状態で結果を得ることである.

\section{手法}

本論文で用いられる手法は,峠の定理と Talenti 関数である.
Hénon 型方程式の係数である $\lvert x \rvert^\alpha$ は
$\Omega$ 上で最大値を取らない.そこで以下の関数を導入し,
Chapter 2 で積分の評価を議論する.
\[
  u_{\epsilon, l}(x) = \frac{\xi_l(x)}{\left(\epsilon +
    \left\lvert x - x_l \right\rvert^2 \right)^{(N-2)/2}}.
\]
ここで, $\epsilon > 0$, $x_l = (1-l, 0, \dots, 0) \in \R^N$ であり,
$\xi_l \in C^\infty_c (\Omega)$ は
$B(x_l, l)$ を台とする切り取り関数である.
$l = l(\epsilon)$ は,
$l \to 0$ ($\epsilon \to 0$) を充たす
$\epsilon$ の関数とみなす.
定理~\ref{thm:henon_main}, ~\ref{thm:kirchhoff_main} を示すために,
次の $2$ 種類の関数を導入する:
$l = l(\epsilon) = \epsilon^\gamma$ ($0 < \gamma < 1/2$),
$l = l(\epsilon) = \lvert \log \epsilon \rvert^{-k}$ ($k > 0$).
前者は既に~\cite{MR2951742} と~\cite{MR2951722}
で,固定された $\gamma$ について考察されているが,
本論文では $\Psi$ は $\partial \Omega$ で消える可能性がある関数であるため,
$\gamma$ と $k$ をうまく選ぶ必要がある.

\eqref{eq:kirchhoff_prob_main} の考察をする際に,
汎関数 $I \colon H_0^1(\Omega) \to \R$ を導入し,峠の定理を適用する.
また $I$ の $(\text{PS})_c$ 条件が達成される $c > 0$ の上界を考察する.
\eqref{eq:kirchhoff_prob_main} を考察する上で困難な点は,
$p$, $N$ が $p = 4$, $N = 3$ とは限らないことである.
~\cite{MR3278854} においては, $p = 4$, $2^* = 6$
であるため, $2$ 次方程式を直接解くことにより
$c$ の上界が得られ,
$2$ 次不等式を直接解くことで, $I$ の峠の高さを評価している.
一方,本論文では,以下の明示的には解けない方程式を考察することになる.
\[
  \left( a + b \eta^{(p-2)/2} \right) \eta - \left( \frac{\eta}{S} \right)^{2^*/2} = 0.
\]
この唯一の正値解を $\eta_0$ とおき,
$\eta_0$ を用いて $c$ の上界の記述を試み,
$I$ の峠の高さの評価を試みる.
このような議論は~\cite{MR3539075} で既に導入されている.
しかし本論文では,
Lions の第 2 集散コンパクト性補題~\cite{MR834360} を含む
~\cite{MR3278854} の議論の直接の拡張を試みるので,
$a = 0$, $b > 0$ の場合を結果に含む.
またこの議論は Hénon 型の係数とも相性が良い.

系~\ref{cor:cor_of_main_theorem} は
定理~\ref{thm:henon_main},~\ref{thm:kirchhoff_main}
の帰結であるが,
\eqref{eq:Psi_0} で定義される $\Psi_0$ の存在を,
$2$ 次元平面への射影を用いて,初等幾何で示す.